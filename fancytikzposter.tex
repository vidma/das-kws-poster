% fancytikzposter.tex, version 2.1
% Original template created by Elena Botoeva [botoeva@inf.unibz.it], June 2012
% 
% This file is distributed under the Creative Commons Attribution-NonCommercial 2.0
% Generic (CC BY-NC 2.0) license
% http://creativecommons.org/licenses/by-nc/2.0/ 

\documentclass{a0poster}

%\usepackage[T1]{fontenc}
%\usepackage[latin9]{inputenc}
\usepackage{fancytikzposter} 


%%%%% --------- Change here if you want ---------- %%%%%
%% margin for the geometry package, must be changed before using the geometry package
%% default value is 4cm
\setmargin{1.2}

%% the space between the blocks
%% default value is 2cm
% \setblockspacing{2}

%% the height of the title stripe in block nodes, decrease it to save space
%% default value is 3cm
% \setblocktitleheight{3}

%% the number of columns in the poster, possible values 2,3
%% default value is 2
% \setcolumnnumber{3}

%% the space between two or more groups of authors from different institutions
%% used in \maketitle
% \setinstituteshift{10}

%% which template to use
%% N1 simple, standard look, with a colored background and gray boxes
%% N2 board with nodes
%% N3 another standard look
%% N4 envelope-like look
%% N5 with a wave-like head, original idea taken from
%%%% http://fc09.deviantart.net/fs71/f/2010/322/1/1/scientific_poster_by_nabuy-d333ria.jpg
%\usetemplate{4}
\usetemplate{2}


%% components of the templates
%% (the maximal possible numbers are mentioned as the parameters)
% \usecolortemplate{4}
% \usebackgroundtemplate{4}
%\usetitletemplate{2}
%\usetitletemplate{2}

%\useblocknodetemplate{4}
% \useplainblocktemplate{4}
% \useinnerblocktemplate{2}


%% the height of the head drawing on top 
%% applicable to templates N3, 4 and 5
% \setheaddrawingheight{14}


%% change the basic colors
%\definecolor{myblue}{HTML}{008888} 
%\setfirstcolor{myblue}% default 116699
%\setsecondcolor{gray!80!}% default CCCCCC
%\setthirdcolor{red!80!black}% default 991111

%% change the more specific colors
% \setbackgrounddarkcolor{colorone!70!black}
% \setbackgroundlightcolor{colorone!70!}
% \settitletextcolor{textcolor}
% \settitlefillcolor{white}
% \settitledrawcolor{colortwo}
% \setblocktextcolor{textcolor}
% \setblockfillcolor{white}
% \setblocktitletextcolor{colorone}
% \setblocktitlefillcolor{colortwo} %the color of the border
% \setplainblocktextcolor{textcolor}
% \setplainblockfillcolor{colorthree!40!}
% \setplainblocktitletextcolor{textcolor}
% \setplainblocktitlefillcolor{colorthree!60!}
% \setinnerblocktextcolor{textcolor}
% \setinnerblockfillcolor{white}
% \setinnerblocktitletextcolor{white}
% \setinnerblocktitlefillcolor{colorthree}



%%% size of the document and the margins
%% A0
% \usepackage[margin=\margin cm, paperwidth=118.9cm, paperheight=84.1cm]{geometry} 
\usepackage[margin=\margin cm, paperwidth=84.1cm, paperheight=118.9cm]{geometry}
%% B1
% \usepackage[margin=\margin cm, paperwidth=70cm, paperheight=100cm]{geometry}



%% changing the fonts
\usepackage{cmbright}
%\usepackage[default]{cantarell}
%\usepackage{avant}
%\usepackage[math]{iwona}
\usepackage[math]{kurier}
\usepackage[T1]{fontenc}
\usepackage[utf8]{inputenc}



%% add your packages here
\usepackage{hyperref}
\usepackage{xcolor}
\usepackage{graphicx}

%\usepackage{subfiles}
\usepackage{standalone}


\usepackage{color}
\usepackage{wrapfig}
\usepackage{graphicx}
\usepackage{nomencl}

\title{Keyword Search over Data Service Integration for Accurate Results}
%\qquad version 2.1
\author{Vidmantas Zemleris\\
%  Faculty of Mathematics and Informatics\\
   Vilnius University, Lithuania\\
   %(at CMS Experiment, CERN)\\
  \texttt{vidmantas.zemleris@cern.ch}\\
  on behalf of CMS Collaboration
  \And
  Valentin Kuznetsov\\
  Cornell University, USA\\
  \texttt{vkuznet@gmail.com}
  \newline
}



%%%%%%%%%%%%%%%%%%%%%%%%%%%%%% LyX specific LaTeX commands.
%% The greyedout annotation environment

\usepackage{color}
\definecolor{note_fontcolor}{rgb}{0.5, 0.5, 0.5}
\definecolor{grey_dark}{rgb}{0.7, 0.7, 0.7}
\definecolor{grey_very_dark}{rgb}{0.3, 0.3, 0.3}
\definecolor{grey_light}{rgb}{0.9, 0.9, 0.9}

\newenvironment{lyxgreyedout}
  {\textcolor{note_fontcolor}\bgroup\ignorespaces}
  {\ignorespacesafterend\egroup}
%% A simple dot to overcome graphicx limitations
\newcommand{\lyxdot}{.}

\usepackage[english,british]{babel}

\usepackage{textcomp}

\begin{document}

\newcommand{\note}{}

%\newcommand{\skip1@preamble}{%
% \let\document\relax\let\enddocument\relax%
% \let\usepackage\relax%
% \newenvironment{document}{}{}%
% \renewcommand{\documentclass}[2][subfiles]{}}
% \newcommand\subfile[1]{\begingroup\skip1@preamble\input{#1}\endgroup}


%%%%% ---------- the background picture ---------- %%%%%
%% to change it modify the macro \BackgroundPicture
\ClearShipoutPicture
\AddToShipoutPicture{\BackgroundPicture}

\noindent % to have the picture right in the center
\begin{tikzpicture}
  \initializesizeandshifts
  % \setxshift{15}
  \setyshift{1.5}


  %% the title block, #1 - shift, the default value is (0,0), #2 - width, #3 - scale
  %% the alias of the title block is `title', so we can refer to its boundaries later
  \ifthenelse{\equal{\template}{1}}{ 
    \titleblock{70}{1.0}
  }{
    %\titleblock{47}{1.5}	
        \titleblock{75}{1.0}

  }

  %% a logo can be added to the title block
  %% #1 - anchor relative to the title block, #2 - shift, #3 - width, #3 - file name
  % \ifthenelse{\equal{\template}{2}}{ 
  %   \addlogo[south west]{(2,0)}{6cm}{unibz_b.png}
  % }{
  %   \addlogo[south west]{(2,0)}{6cm}{unibz_w.png}
  % }
  \addlogo[south west]{(0,-1.5)}{6cm}{images/vu_logo.png}
  \addlogo[south west]{(7,-1.5)}{6cm}{images/mif.png}
	%
  \addlogo[south east]{(0,-1.5)}{6cm}{images/cern_logo.png}
  \addlogo[south east]{(-7,-1.5)}{6cm}{images/cms_logo.png}



  %% a block node, with the specified position (optional), title and the content
  %% #1 - where (optional), #2 - title, #3 - text
  %%%%%%%%%% ------------------------------------------ %%%%%%%%%%
%  \getcurrentrow{box}
  %\blocknodew[(0, 45)]{75}%
  \blocknode%
  {Summary}%
  {%
  %\textbf{Background:} 
  Virtual data integration aims at providing a coherent interface for querying heterogeneous data sources (e.g. web services, proprietary systems) with minimum upfront effort in integration. %
  %This work explores various aspects of its usability.
Data is usually accessed through structured queries, such as SQL, requiring to learn the language and to get acquainted with data organization, which may pose problems even to proficient users.
% thus negatively impacting the system’s usability.%

%\note{, given a keyword query,}
%\textbf{Solution:} 
\quad We present a keyword search system, which proposes a ranked list of structured queries along with their explanations. It operates mainly on the metadata, such as the constraints on inputs accepted by services.
%{\color{red}, analysis of user queries, and only certain portions of the data.} %
% nlike previous implementations,
% and makes no assumptions about the input query, while maintaining its ability to leverage the query’s structural patterns - in case they exist.}%
It was developed as an integral part of the CMS data discovery service 
	% {\color{red} focusing on simplicity and capabilities of the interface},
	 and is currently available as open source.
%{\color{red}The system is }.
}


  %% a callout block
  %% #1 - rotate angle (optional), #2 - from, #3 - where, #4 - width, #5 - text
  %%%%%%%%%% ------------------------------------------ %%%%%%%%%%
%  \calloutblock{($(box.center)+(-2,-8)$)}
%  {($(box.center)+(10,-1)$)}
%  {19cm}
%  {\small
%    Macro for creating a block node:
%    \begin{itemize}
%    \item[] \textbackslash blocknode\{Block Title\}\{Block Content\}
%    \end{itemize}
%    Macro \textbackslash blocknode has three parameters. The first one is
%    optional and it is the position of the block. The first block will be
%    automatically placed to (\$(firstrow)-(xshift)-(yshift)\$), which is the
%    left corner below the title block. In most of the templates, (firstrow) is
%    set to (title.south), where \emph{title} is the alias for the title
%    block. Each subsequent block is automatically placed to
%    [(\$(box.south)-(yshift)\$)], i.e., below the previous block aliased
%    \emph{box}.  You can also use an explicit parameter, e.g., $(-10,30)$ (note
%    that (0,0) is the center of the poster). The second parameter is the title
%    of the block. Finally, the last parameter is the  actual content. 
%  }




  %% by default, the position of the new block node is right below the previous
  %% block node, stored in (currenty)
  %% box is the alias of the previous block, so we can refer to its boundaries

  %%%%%%%%%% ------------------------------------------ %%%%%%%%%%
  \blocknode{Context: a system for Virtual Data Integration}%
  {%
  %% LyX 2.0.6 created this file.  For more info, see http://www.lyx.org/.
%% Do not edit unless you really know what you are doing.
\documentclass{standalone}
\usepackage{graphicx}

\makeatletter

%%%%%%%%%%%%%%%%%%%%%%%%%%%%%% LyX specific LaTeX commands.
%% A simple dot to overcome graphicx limitations
\newcommand{\lyxdot}{.}


\makeatother

\begin{document}
\begin{itemize}
\item uses simple structured queries 
\item integrated access to proprietary data-sources 

\begin{itemize}
\item process the query \& send requests to services 
\item eliminates inconsistencies in entity naming, data formats(XML, JSON)
\item combining the results
\end{itemize}
\item uses lightweight service mappings

\begin{itemize}
\item minimal effort in defining services
\item complete services structure is figured out in runtime
\end{itemize}
\end{itemize}

\paragraph{Query language and Execution flow}

The queries are formed specifying the entity the user is interested
in (e.g. dataset, file, etc) and providing filtering criteria (e.g.
attribute=value, attribute \emph{between} {[}v1, v2{]}). The results
could be later 'piped' for further filtering, sorting or aggregation
(min, max, avg, sum, count, median), e.g.: 

\includegraphics[width=1\textwidth]{/home/vidma/DAS_paper/figures/DASQL_structure}
\end{document}
%
  }
  \plainblock[3]{($(box.south)+(3,1)$)}{30}%
  {\vspace{-20pt}\newline%
    still, it is overwhelming for users to:}%
  {%\vspace{-10pt}
  \begin{itemize}
  	\item learn the query language
	\item remember how exactly the data is structured and named 
  \end{itemize}
  \textbf{\textit{Could keyword queries solve this?}}%
  \vspace{-20pt}}
  
  \getcurrentrow{note}
  \coordinate (currenty) at ($(currentrow)-(yshift)-(xshift)+(0,0.5)$);


  %%%%%%%%%% ------------------------------------------ %%%%%%%%%%
  \blocknode{Interpreting Keyword Queries: Problem definition} %
  {%
    %% LyX 2.0.6 created this file.  For more info, see http://www.lyx.org/.
%% Do not edit unless you really know what you are doing.
\documentclass{standalone}
\usepackage{wrapfig}
\usepackage{graphicx}
\usepackage{nomencl}
% the following is useful when we have the old nomencl.sty package
\providecommand{\printnomenclature}{\printglossary}
\providecommand{\makenomenclature}{\makeglossary}
\makenomenclature

\makeatletter

%%%%%%%%%%%%%%%%%%%%%%%%%%%%%% LyX specific LaTeX commands.
%% A simple dot to overcome graphicx limitations
\newcommand{\lyxdot}{.}


\makeatother

\begin{document}
\begin{wrapfigure}{o}{0.4\textwidth}%
\vspace{-40pt}

\includegraphics[width=0.4\textwidth]{/home/vidma/DAS_paper/figures/problem_statement_das_service_ex}\vspace{-20pt}\caption{a data-service (simplified)}
\end{wrapfigure}%


\textbf{Input:} query, KWQ\nomenclature{KWQ}{keyword query\\}=$\left(kw_{1},kw_{2},..,kw_{n}\right)$

\textbf{Task: }translate it into structured query

\textbf{Given: }metadata
\begin{itemize}
\item names of entities and their attributes 

\begin{itemize}
\item (either \emph{service inputs} or their \emph{output} fields)
\end{itemize}
\item possible values (only for some inputs)
\item \emph{constraints} on data-service \emph{inputs}:

\begin{itemize}
\item mandatory inputs
\item regular expressions on values\end{itemize}
\end{itemize}

\end{document}

  }
  %%%%%%%%%% ------------------------------------------ %%%%%%%%%%

  \setplainblockfillcolor{grey_light}
  \calloutblock[4]{($(box.center)+(10,6)$)}%
  		{($(box.south east)+(-11,10)$)}{23} %
  {%    
    \vspace{0.3cm}
    %% LyX 2.0.6 created this file.  For more info, see http://www.lyx.org/.
%% Do not edit unless you really know what you are doing.
\documentclass{standalone}
\usepackage{color}
\usepackage{graphicx}

\makeatletter

%%%%%%%%%%%%%%%%%%%%%%%%%%%%%% LyX specific LaTeX commands.
%% A simple dot to overcome graphicx limitations
\newcommand{\lyxdot}{.}


\makeatother

\begin{document}
\textbf{Example. }Consider this query: average\textcolor{red}{{} }size
of RelVal datasets with its number of events \textgreater{} 1000
\begin{itemize}
\item average RelVal dataset size nevents\textgreater{}1000
\item avg(dataset size) RelVal ``number of events''\textgreater{}1000
\end{itemize}
For all, the expected result is:

\includegraphics[width=1\textwidth]{/home/vidma/DAS_paper/figures/DASQL_1avg_problem_statement}


\end{document}
%
    }
  %% the coordinate (currenty) is used in the default placing of the next blocknode
  %\getcurrentrow{note}
  \getcurrentrow{note}
  %\coordinate (currenty) at ($(currentrow)-(yshift)-(xshift)+(0,2)$);
  \coordinate (currenty) at ($(currentrow)-(yshift)-(xshift)+(0,0)$);


  %%%%%%%%%% ------------------------------------------ %%%%%%%%%%
  \blocknode%
  {Keyword search overview}%
  {
	%% LyX 2.0.6 created this file.  For more info, see http://www.lyx.org/.
%% Do not edit unless you really know what you are doing.
\documentclass{standalone}
\usepackage{color}
\definecolor{note_fontcolor}{rgb}{0.80078125, 0.80078125, 0.80078125}
\usepackage{graphicx}

\makeatletter

%%%%%%%%%%%%%%%%%%%%%%%%%%%%%% LyX specific LaTeX commands.
%% The greyedout annotation environment
\newenvironment{lyxgreyedout}
  {\textcolor{note_fontcolor}\bgroup\ignorespaces}
  {\ignorespacesafterend\egroup}
%% A simple dot to overcome graphicx limitations
\newcommand{\lyxdot}{.}


\makeatother

\begin{document}
\selectlanguage{english}%
\begin{center}
\begin{minipage}[t]{0.5\columnwidth}%
\begin{itemize}
\item \emph{tokenizer}: 

\begin{itemize}
\item clean up the query
\item identify patterns
\end{itemize}
\item identify and score ``\emph{entry points}'' with

\begin{itemize}
\item string matching %
\begin{lyxgreyedout}
{[}for entity names{]}%
\end{lyxgreyedout}

\item IR (IDF-based)%
\begin{lyxgreyedout}
 {[}output fieldnames{]}%
\end{lyxgreyedout}

\item list of known values
\item regular expressions on allowed values
\end{itemize}
\item combine\emph{ entry points}

\begin{itemize}
\item consider various \emph{entry point} permutations %
\begin{lyxgreyedout}
 (keyword labelings)%
\end{lyxgreyedout}

\item promote ones respecting keyword dependencies or other heuristics
\item interpret as structured queries \selectlanguage{english}%
\end{itemize}
\end{itemize}
%
\end{minipage}\,%
\begin{minipage}[t]{0.4\columnwidth}%
\selectlanguage{british}%
\textbf{\vspace{-80}}

\textbf{\includegraphics[width=0.99\textwidth]{/home/vidma/DAS_paper/figures/DAS_KWS_architecture}}\selectlanguage{english}%
%
\end{minipage} 
\par\end{center}\selectlanguage{british}%

\end{document}

  }
  %% to place the next node centered vertically in the second column, we can
  %% obtain the y-coordinate of the previous node using macro
  %% \getcurrentrow{note}, where note is the alias of the callout node, and
  %% then specify the coordinate of the next node using coordinate (currentrow)
  \getcurrentrow{box} 
  \coordinate (kws_north) at ($(box.north)$);
  \coordinate (kws_overv) at ($(box.south east)$);  

  \setplainblockfillcolor{white}
  \setplainblocktitlefillcolor{grey_dark}
  \renewcommand{\plainblocktemplate}{3}
  
  \plainblock[0]{($(box.south west)-(yshift)+(3,0)$)}%
  {6.5cm}{} %
  {\newline\vspace{-1.8cm}\newline
  	\textbf{More info:}\vspace{0.5cm}
  	\newline
    \includegraphics[width=5cm]{images/qrcode.png}
  	\vspace{-1.5cm}
  }  
  % restore plain block template
  \renewcommand{\plainblocktemplate}{2}
  
  
  %% a plain block
  %% #1 - rotate angle (optional), #2 - where, #3 - width, #4 - title, #5 - text
  %%%%%%%%%% ------------------------------------------ %%%%%%%%%%
  \setplainblockfillcolor{white}
  \setplainblockfillcolor{grey_light}
  
  \calloutblock[0]{($(kws_overv)+(-4.9,2.3)$)}%
  		{($(box.south east)-(yshift)+(-14, 0)$)} %
  {27} %
  {
  %\textbf{\color{blue}Result presentation:}
  %\newline
  	\includegraphics[width=1.0\textwidth]{ui_result.png}
  	\vspace{-50pt}
  }
  
  \getcurrentrow{note}
  \coordinate (currenty) at ($(currentrow)+(yshift)$);

  %% the coordinate (currenty) is used in the default placing of the next blocknode
 %\getcurrentrow{note}
 %\coordinate (currenty) at ($(currentrow)+(xshift)-(yshift)$);





  %%%%%%%%%%%%% NEW COLUMN %%%%%%%%%%%%%%% 
  \startsecondcolumn 

  %%%%%%%%%% ------------------------------------------ %%%%%%%%%%
  \blocknode%
  {Challenges}%
  {
	%% LyX 2.0.6 created this file.  For more info, see http://www.lyx.org/.
%% Do not edit unless you really know what you are doing.
\documentclass{standalone}
\usepackage{color}
\definecolor{note_fontcolor}{rgb}{0.80078125, 0.80078125, 0.80078125}
\usepackage{textcomp}

\makeatletter

%%%%%%%%%%%%%%%%%%%%%%%%%%%%%% LyX specific LaTeX commands.
%% The greyedout annotation environment
\newenvironment{lyxgreyedout}
  {\textcolor{note_fontcolor}\bgroup\ignorespaces}
  {\ignorespacesafterend\egroup}

\makeatother

\begin{document}
\begin{itemize}
\item queries are ambiguous

\begin{itemize}
\item return ranked list of structured query suggestions
\end{itemize}
\item no direct access to the data

\begin{itemize}
\item querying services is expensive \textrightarrow{} rely on metadata
\item bootstrap list of allowed values %
\begin{lyxgreyedout}
(available only for some fields) %
\end{lyxgreyedout}

\item rely on \emph{regexps} with lower confidence %
\begin{lyxgreyedout}
(can result in false positives)%
\end{lyxgreyedout}

\end{itemize}
\item no predefined schema

\begin{itemize}
\item bootstrap list of fields in service results through queries
\item some field names are unclean \textrightarrow{} use IDF %
\begin{lyxgreyedout}
(as they come directly from JSON/XML responses)%
\end{lyxgreyedout}
\end{itemize}
\end{itemize}

\end{document}

  }


  %%%%%%%%%% ------------------------------------------ %%%%%%%%%%
  

  \setinnerblocktitlefillcolor{grey_dark}

  \blocknode%
  {The ranker}%
  {
	%% LyX 2.0.6 created this file.  For more info, see http://www.lyx.org/.
%% Do not edit unless you really know what you are doing.
\documentclass{standalone}
\usepackage{color}
\definecolor{note_fontcolor}{rgb}{0.80078125, 0.80078125, 0.80078125}

\makeatletter

%%%%%%%%%%%%%%%%%%%%%%%%%%%%%% LyX specific LaTeX commands.
%% The greyedout annotation environment
\newenvironment{lyxgreyedout}
  {\textcolor{note_fontcolor}\bgroup\ignorespaces}
  {\ignorespacesafterend\egroup}

\makeatother

\begin{document}
\selectlanguage{english}%
\[
score\_prob=\sum_{i=1}^{|KWQ|}\left({\displaystyle {\displaystyle \ln}\left(score(tag_{i}|kw_{i})\right)}+\sum_{h_{j}\in H}h_{j}(tag_{i}|kw_{i};tag_{i-1,..,1})\right)
\]

\begin{itemize}
\item $score(tag_{i}|kw_{i})$ - likelihood of $kw_{i}$ to be $tag_{i}$
(from entry points step) 
\item $h_{j}(tag_{i}|kw_{i};tag_{i-1,..,1})$ - the score boost returned
by heuristic $h_{j}$ given a tagging so far (often all $i-1$ tags
are not needed). 
\end{itemize}
\selectlanguage{british}%

\paragraph{\textsc{So far we use exhaustive search ranker. why?}}
\begin{itemize}
\item simpler; our schema is quite small

\begin{itemize}
\item \textcolor{red}{\small{our}}\textbf{\textcolor{red}{\small{ cython}}}\textcolor{red}{\small{
implementation is quite fast (often bound by MongoDB/IR engine to
retrieve entry points)}}{\small \par}
\end{itemize}
\item allows finding optimal solutions
\item \textcolor{red}{easy to filter out MANY ``invalid'' solution candidates}
\begin{lyxgreyedout}
that are not supported by services yet%
\end{lyxgreyedout}


\begin{itemize}
\item large amounts of data managed by services; performance considerations\end{itemize}
\end{itemize}

\end{document}

  }
  % set it back to what it was
  %\setinnerblocktitlefillcolor{colorthree}



  

  %%%%%%%%%% ------------------------------------------ %%%%%%%%%%
  \blocknode {Related works}%
  {
  	%% LyX 2.0.6 created this file.  For more info, see http://www.lyx.org/.
%% Do not edit unless you really know what you are doing.
\documentclass{standalone}
\usepackage{color}
\definecolor{note_fontcolor}{rgb}{0.80078125, 0.80078125, 0.80078125}

\makeatletter

%%%%%%%%%%%%%%%%%%%%%%%%%%%%%% LyX specific LaTeX commands.
%% The greyedout annotation environment
\newenvironment{lyxgreyedout}
  {\textcolor{note_fontcolor}\bgroup\ignorespaces}
  {\ignorespacesafterend\egroup}

\makeatother

\begin{document}

\paragraph*{\vspace{-5cm}}


\paragraph{}
\begin{itemize}
\item Keymantic (the closest work)

\begin{enumerate}
\item score keyword mappings individually %
\begin{lyxgreyedout}
 (entry points)%
\end{lyxgreyedout}
{} 
\item solve\emph{ ``weighted bipartite assignment'' }($kw_{i}\rightarrow tag_{j}$)
\emph{with contextualizations:}

\begin{itemize}
\item maximize total sum of weights
\item uses heuristics to account for keyword interdependencies (contextualization)

\begin{itemize}
\item {\small{e.g. \textless{}table\_name\textgreater{} \textless{}attribute\textgreater{};
\textless{}attribute\textgreater{} \textless{}its value\textgreater{}; }}{\small \par}
\item {\small{solves it }}\emph{\small{approximately}}{\small{ with Munkres
algorithm modified to consider}}\emph{\small{ contextualizations:}}{\small \par}

\begin{itemize}
\item {\small{contextualize - modify weights of $\ensuremath{kw_{i}\rightarrow tag_{j}}$,
if $tag_{j}$ is ``related'' to earlier sub-assignments}}{\small \par}
\item {\small{to get multiple results, repeat recursively forcing/preventing
certain sub-assignments}}{\small \par}
\item \textcolor{red}{\small{Presenter's note: may not catch the optimum
solution if:}}{\small \par}

\begin{itemize}
\item \textcolor{red}{\small{contextualization plays crucial role in selecting
it}}{\small \par}
\item \textcolor{red}{\small{and do not wish to generate ALL assignments}}{\small \par}
\end{itemize}
\end{itemize}
\end{itemize}
\end{itemize}
\item interpret generated mappings as SQL queries
\end{enumerate}
\item KEYRY - uses HMM %
\begin{lyxgreyedout}
(Hidden Markov Model) %
\end{lyxgreyedout}
to label keywords as schema terms

\begin{itemize}
\item {\small{HMM's initial parameters can be estimated from similar heuristics
as above}}{\small \par}
\item {\small{later machine learning can be used }}%
\begin{lyxgreyedout}
{\small{(if logs available)}}%
\end{lyxgreyedout}
{\small \par}
\end{itemize}
\end{itemize}

\paragraph{\textsc{}}
\end{document}

  }
  
  \getcurrentrow{box}   

  %%%%%%%%%% ------------------------------------------ %%%%%%%%%%

  \renewcommand{\plainblocktemplate}{3}

  \setplainblockfillcolor{grey_light}
  % this is also for arrow borders
  %\setplainblocktitlefillcolor{grey_dark}
  
  \calloutblock[0]{($(kws_north)+(14,-1.0)$)}%
  		{($(box.south)-(yshift)-(5,0)$)}{30} %
  {%    
    \vspace{0.3cm}
    \textbf{Autocompletion to ease typing the queries (prototype)}\newline
     \includegraphics[width=28cm]{images/autocompl.png}
         \vspace{-1cm}

    }
  %% the coordinate (currenty) is used in the default placing of the next blocknode
  %\getcurrentrow{note}
  \getcurrentrow{note}
  
  \coordinate (currenty) at ($(currentrow)-(0, 1.5)+(xshift)$);

  %%%%%%%%%% ------------------------------------------ %%%%%%%%%%


  \calloutblock[0]{($(kws_north)+(14,-4.5)$)}%
  		{($(currenty)-(7.5,-0.5)$)}{25} %
  {%    
    %\vspace{-0.3cm}
    \newline
    \textbf{Tokenized query:} 'relval', 'number', 'of', 'events>100'
    \newline%
    \vspace{0.2cm}
    }

  \getcurrentrow{note}
  \coordinate (currenty) at ($(currentrow)-(0, 0.5)+(xshift)$);

  %%%%%%%%%% ------------------------------------------ %%%%%%%%%%


  \calloutblock[0]{($(kws_north)+(15.5,-13.5)$)}%
  		{($(currenty)-(7.5,0)$)}{25} %
  {%    
\vspace{-0.5cm}
    \textbf{Entry points:}\newline
    \input{./entry_points}
        \vspace{-0.3cm}

    }

  \getcurrentrow{note}
  \coordinate (currenty) at ($(currentrow)-(0, 1)+(xshift)$);


  %%%%%%%%%% ------------------------------------------ %%%%%%%%%%
  \blocknode {Future work}%
  {
    %% LyX 2.0.6 created this file.  For more info, see http://www.lyx.org/.
%% Do not edit unless you really know what you are doing.
\documentclass{standalone}
\usepackage{color}
\definecolor{note_fontcolor}{rgb}{0.80078125, 0.80078125, 0.80078125}

\makeatletter

%%%%%%%%%%%%%%%%%%%%%%%%%%%%%% LyX specific LaTeX commands.
%% The greyedout annotation environment
\newenvironment{lyxgreyedout}
  {\textcolor{note_fontcolor}\bgroup\ignorespaces}
  {\ignorespacesafterend\egroup}

\makeatother

\begin{document}

\paragraph*{\vspace{-3cm}}
\begin{itemize}
\item \textbf{advanced autocompletion {[}photo!{]}}
\item improve the ranker
\item look into advanced methods of generic performance improvements to
the services like ``materialized view with the incremental refresh''
\end{itemize}

\paragraph{\textsc{top-K (semi-)optimal solutions with contextualization?}}

\begin{lyxgreyedout}
(only ideas for now...):%
\end{lyxgreyedout}

\begin{itemize}
\item maybe Ranked (Murty\textquoteright{}s) Munkres with Contextualization!?

\begin{itemize}
\item this would at least guarantee optimal top-k for with \textbf{some}
contextualization
\end{itemize}
\item \textbf{out of scope, ask for handouts}
\end{itemize}

\paragraph{\textsc{Reflections on the HMM approach:}}
\begin{itemize}
\item what is modelled is not same as seen by user

\begin{itemize}
\item models $kw_{i}\rightarrow t_{j},$ while user sees structured queries
\item therefore, hard to get automatically collect training data \end{itemize}
\end{itemize}

\end{document}

  }



%  \plainblock[0]{($(box.south east)-(yshift)+(-5,0)$)}%
%  {8cm}{} %
%  {\newline\vspace{-0cm}
%  	\textbf{More info:}\vspace{0.5cm}
%  	\newline
%    \includegraphics[width=5cm]{images/qrcode.png}
%  	\vspace{-0cm}
%  }  
%    


\end{tikzpicture}


\end{document}




